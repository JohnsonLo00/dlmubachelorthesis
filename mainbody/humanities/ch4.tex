
\section{系统性能分析}

{\bf 伪代码的使用示例如下所示。}

\begin{algorithm}
\SetKwData{Left}{left}\SetKwData{This}{this}\SetKwData{Up}{up} % 定义快捷变量,后续使用时只需:反斜杠+变量名
\SetKwFunction{Union}{Union}\SetKwFunction{FindCompress}{FindCompress} % 定义函数,后续使用时只需:反斜杠+函数名
\SetKwInOut{Input}{输入}\SetKwInOut{Output}{输出} %此处可自定义输入、输入的名称格式

在此处添加不带编号的内容(若无,则将该行注释即可)。\\
\LinesNumbered % 使下列算法描述带行号
\Input{A bitmap $Im$ of size $w\times l$} % 输入
\Output{A partition of the bitmap} % 输出
\BlankLine

\emph{special treatment of the first line}\;
\For{$i\leftarrow 2$ \KwTo $l$}{
  \emph{special treatment of the first element of line $i$}\;
  \For{$j\leftarrow 2$ \KwTo $w$}{\label{forins}
    \Left$\leftarrow$ \FindCompress{$Im[i,j-1]$}\;
    \Up$\leftarrow$ \FindCompress{$Im[i-1,]$}\;
    \This$\leftarrow$ \FindCompress{$Im[i,j]$}\;
    \If( \tcp*[h]{此处添加注释:O(\Left,\This)==1} ){\Left compatible with \This}{\label{lt} % \tcp*[h]{...}表示非对齐的注释
      \lIf{\Left $<$ \This}{\Union{\Left,\This}} %含else...的if语句用命令\lIf
      \lElse{\Union{\This,\Left}}
    }
    \If(\tcp*[f]{O(\Up,\This)==1}){\Up compatible with \This}{\label{ut} % \tcp*[f]{...}表示居右对齐的注释
      \lIf(\tcp*[f]{再次用Union函数}){\Up $<$ \This}{\Union{\Up,\This}}
      \tcp{\This is put under \Up to keep tree as flat as possible}\label{cmt} % \tcp{...}表示行内的注释
      \lElse{\Union{\This,\Up}}\tcp*[h]{\This linked to \Up}\label{lelse}
    }
  }
  \lForEach{element $e$ of the line $i$}{\FindCompress{p}}
}
\caption{不相交分解(disjoint decomposition)}\label{pcode: 不相交分解} %伪代码对应算法的标题及其引用标签
\end{algorithm}

引用时的格式为:算法\ref{pcode: 不相交分解}实现了...。

{\bf 下面用于展示基于\verb|changes|宏包的批注功能。}

北冥有鱼,其名为鲲。鲲之大,不知其几千里也;化而为鸟,其名为鹏。鹏之背,不知其几千里也;怒而飞,其翼若垂天之云。\added[id=海哥, comment={少了一句}]{是鸟也,海运则将徙于南冥。}南冥者,天池也。《齐谐》者,志怪者也。《谐》之言曰:“鹏之徙于南冥也,水击三千里,抟扶摇而上者九万里,去以六月息者也。”野马也,尘埃也,生物之以息相吹也。\deleted[id=海哥, comment={这句话删掉}]{之乎者也。}天之苍苍,其正色邪?其远而无所至极邪?其视下也,亦若是则已矣。且夫水之积也不厚,则其负大舟也无力。覆杯水于\replaced[id=海老, comment={用错词}]{坳堂}{水堂}之上,则芥为之舟,置杯焉则胶,水浅而舟大也。风之积也不厚,则其负大翼也无力。故九万里,则风斯在下矣,而后乃今培风;背负青天,而莫之夭阏者,而后乃今将图南。蜩与学鸠笑之曰:“我决起而飞,抢榆枋而止,时则不至,而控于地而已矣,奚以之九万里而南为?”适莽苍者,三餐而反,腹犹果然;适百里者,宿舂粮;适千里者,三月聚粮。之二虫又何知!

小知不及大知,小年不及大年。奚以知其然也?朝菌不知晦朔,蟪蛄不知春秋,此小年也\deleted[id=海老]{,不亦乐乎}。楚之南有冥灵者,以五百岁为春,五百岁为秋;上古有大椿者,以八千岁为春,八千岁为秋,此大年也。而彭祖乃今以久特闻,众人匹之,不亦悲乎!汤之问棘也是已。穷发之北,有冥海者,天池也。有鱼焉,其广数千里,未有知其修者,其名为鲲。有鸟焉,其名为鹏,背若泰山,翼若垂天之云,抟扶摇羊角而上者九万里,绝云气,负青天,然后图南,且适南冥也。斥鴳笑之曰:“彼且奚适也?我腾跃而上,不过数仞而下,翱翔蓬蒿之间,此亦飞之至也。而彼且奚适也?”此小大之辩也。

\highlight[id=海老]{故夫知效一官,行比一乡,德合一君,}而征一国者,其自视也,亦若此矣。而宋荣子犹然笑之。且举世誉之而不加劝,举世非之而不加沮,定乎内外之分,辩乎荣辱之境,斯已矣。彼其于世,未数数然也。虽然,犹有未树也。夫列子御风而行,泠然善也,旬有五日而后反。彼于致福者,未数数然也。此虽免乎行,犹有所待者也。\highlight[id=海老, comment={建议对其展开分析}]{若夫乘天地之正,而御六气之辩,以游无穷者,彼且恶乎待哉}?故曰:至人无己,神人无功,圣人无名\comment[id=海老]{建议把落款补上}。

\iffalse
只能用这个命令
来实现多行注释
\fi
