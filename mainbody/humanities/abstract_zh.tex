
% ---------- 中文摘要内容 ---------- %
{\zihao{-4} %【该句涉及到格式设置,请勿乱动】

医疗保障是社会保障体系的重要组成部分。完善的医疗保障体系是人民群众生活的基本需求,是整个社会正常运转的基础,是经济发展的促进剂,所以提高我国的医疗保障水平是十分必要的。本文依据医疗保障的基本理论,分四部分对我国的医疗保障制度和水平做了系统性的研究。第一部分先是介绍了医疗保障和医疗保障水平的含义,然后从医疗保障在社会中发挥的功能、作用和重要地位三方面阐述了对我国医疗保障水平进行研究的必要性。第二部分对城镇和农村的医疗保障水平进行了具体地介绍,并将两方面进行分析比较,其中对公平性方面做了着重分析,从而综合性地阐述了我国医疗保障制度的现状。第三部分对国外几个比较有代表性的国家英国、德国、美国、新加坡的医疗保障制度进行介绍,并与我国的医疗保障水平比较后总结出适合我国发展的经验。第四部分就未来如何提高我国的医疗保障水平问题提出一些对策和建议。包括社会医疗保险制度的完善与发展,明确基本医疗保障范围,逐步实现覆盖全民一体化医疗保障体制,政府加强医疗保障投入,建立平价医院五方面。


\keywordsCN{医疗保障水平;医疗保障制度;对策}
}

