

医疗保障制度是社会保障制度的重要组成部分。完善的医疗保障制度,不仅可以有效的维护社会经济安全、稳定的发展,更重要的是它是促进社会公正、维护公民基本权利的一种重要手段。但是长期以来,我国的医疗保障水平依旧存在着一些的问题。城乡差异较大,公平性有待提高等都是当前摆在我们面前亟待解决的问题。

\section{我国医疗保障水平}

\subsection{医疗保障含义}

医疗保障是指一个社会通过正式的和非正式的制度安排为其公民提供所有健康保障服务的一种社会制度。医疗保障分为广义和狭义两种。“只要我想到我是什么东西,他就永远不能使我成为什么都不是”;“那么我究竟是什么呢?是一个在思维的东西。”\footnote{勒内•笛卡尔.第一哲学沉思录.北京:九州出版社,2007:P43}笛卡尔对“人是什么”这个问题的解答很好地诠释了理性主义在其萌发时期,为理性——即“思”树立为人的唯一本质性存在的基础埋下伏笔。

\subsection{医疗保障对社会发展的作用}

\subsubsection{维护社会稳定}

为减少社会风险,保持社会的稳定发展,通过建立包括医疗保障制度在内的社会保障体系来减轻社会风险对社会稳定造成的冲击\upcite{童星2002},使之成为维护社会稳定的一种“减震器”。

\subsubsection{促进经济发展}

医疗保险的社会化管理,提高了医疗保险基金的共济能力\upcite{郑功成2002},使用人单位能够用较少的费用即能达到保障目的,减轻了企业的负担,有利于推进企业改革的深入和企业的成长壮大。

\subsubsection{调节收入分配}

医疗保险缴费与收入挂钩,不同收入人群缴费不一样,通过医疗保险基金发挥共济作用,从纵向实现调节收入分配功能。参保人员享受相同的保障待遇,体现了横向公平。“我思故我在”,作为古典理性主义哲学理论上的出发点,在意志主义哲学中,被解读为:“思”若是条件,则“我”是受制约的;因此,“我”只是一个综合,此综合是由思本身制造的。\footnote{尼采.论道德的谱系,善恶之彼岸.桂林:漓江出版社,2007:P165.}

\subsection{本章小结}

本章主要介绍了医疗保险的相关知识。医疗保障是指一个社会通过正式的和非正式的制度安排为其公民提供所有健康保障服务的一种社会制度。

