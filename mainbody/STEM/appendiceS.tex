
\vspace*{0pt} %【该句涉及到格式设置,请勿乱动】

\pagestyle{plain} %《手册》要求:从附录页开始,去掉页眉

{\zihao{5} %【该句涉及到格式设置,请勿乱动】

\appendixsec{企业信息表}

\begin{longtable}{cccc}
    \caption{企业信息表} \label{tab:企业信息表} \\
	\toprule
	字段名称 & 中文描述 & 类型 & 长度 \\
	\midrule
    \endfirsthead %以上是首页表头
    \caption{(续表)} \\ %续表对应的标题
    \midrule
    字段名称 & 中文描述 & 类型 & 长度 \\
    \midrule
    \endhead %以上是通用表头
    \midrule
    \endfoot %以上是通用表尾
    \bottomrule
    \endlastfoot %以上是末页表尾,以下是表格正文
    ID & ID & NUMBER & 15 \\
    COMPANY\_ID & 公司ID & VARCHAR2 & 60 \\
    LOGISTER\_AGENT & 委托代理人 & VARCHAR2 & 60 \\
    SHORT\_NAME & 物流商简称 & VARCHAR2 & 60 \\
    BUSINESS\_FIELD & 行业类别 & VARCHAR2 & 10 \\
    WAY\_VEHICLE & 公路运输 & VARCHAR2 & 10 \\
    WAY\_TRAIN & 铁路运输 & VARCHAR2 & 10 \\
    WAY\_SHIP & 船舶运输 & VARCHAR2 & 10 \\
    WAY\_PIPELINE & 管道运输 & VARCHAR2 & 10 \\
    WAY\_CONTAINER & 集装箱运输 & VARCHAR2 & 10 \\
    WAY\_OTHERS & 其他运输方式 & VARCHAR2 & 60 \\
    FAX & 传真 & DATE & \\
    SETUP\_DATE & 成立日期 & VARCHAR2 & 60 \\
    BUSINESS\_LICENSECODE & 营业执照号码 & DATE & \_\_ \\
    BUSINESS\_LICENSEDATE & 营业执照有效期 & VARCHAR2 & 60 \\
    GAS\_LICENSECODE & 许可证号码 & DATE & \_\_ \\
    GAS\_LICENSEDATE & 许可证有效期 & VARCHAR2 & 60 \\
    HAZARD\_LICENSECODE & \makecell{化学危险品经营 \\ 许可证号码} & DATE & \_\_ \\
    HAZARD\_LICENSEDATE & \makecell{化学危险品经营 \\ 许可证有效期} & VARCHAR2 & 60 \\
    STATE\_TAXACCOUNT & 国税税号 & VARCHAR2 & 60 \\
    CREATE\_USERID & 创建人 & NUMBER & 15
\end{longtable}

\appendixsec{相关定理证明}

\appendixsec{程序代码}

% 直接输入的形式插入代码

\begin{lstlisting}[language=C++]
int main(){
    int i;
    printf("hello latex!\n");
    return 0;
}
\end{lstlisting}


% 以文件形式插入代码

\lstinputlisting[ language=C++, title={\raggedright\normalsize 计算$\bk{n}$的阶乘(C++):} ]{codes/funfactorial.cpp} %C++

\lstinputlisting[ language=Java, title={\raggedright\normalsize 计算$\bk{n}$的阶乘(Java):} ]{codes/funfactorial.java} %Java

\lstinputlisting[ language=Python, title={\raggedright\normalsize 计算$\bk{n}$的阶乘(Python):} ]{codes/funfactorial.py} %Python

\lstinputlisting[style=Matlab-editor, title={\raggedright\normalsize 计算$\bk{n}$的阶乘(MATLAB):}]{codes/funfactorial.m} %MATLAB(如果不想要这种风格,则把该行命令的可选参数style=Matlab-editor改为language=Matlab

} %字号设置